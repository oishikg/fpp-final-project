\documentclass[12pt, a4paper]{article}

\usepackage{fancyhdr}
\usepackage{mathtools, amssymb}
% \usepackage[mathcal]{eucal}
\usepackage{array, booktabs, multirow}
\usepackage{verbatim}
\usepackage[left=1 in, bottom=1.5 in, headheight=15pt]{geometry}
\usepackage{xcolor}
\usepackage{listings}

\usepackage{color}
\definecolor{codegreen}{rgb}{0,0.6,0}
\definecolor{codegray}{rgb}{0.5,0.5,0.5}
\definecolor{codepurple}{rgb}{0.58,0,0.82}
\definecolor{backcolour}{rgb}{0.95,0.95,0.92}

\lstdefinestyle{mystyle}{
  backgroundcolor=\color{backcolour},   
  commentstyle=\color{codegreen},
  keywordstyle=\color{magenta},
  numberstyle=\tiny\color{codegray},
  stringstyle=\color{codepurple},
  basicstyle=\footnotesize,
  breakatwhitespace=false,         
  breaklines=true,                 
  captionpos=b,                    
  keepspaces=true,                 
  numbers=left,                    
  numbersep=5pt,                  
  showspaces=false,                
  showstringspaces=false,
  showtabs=false,                  
  tabsize=2
}

\lstset{
  language=[Objective]Caml
}

\lstset{style=mystyle}
\setlength{\parskip}{2ex}


\pagestyle{fancy}

\newcommand{\CC}{\mathbb{C}}
\newcommand{\RR}{\mathbb{R}}
\newcommand{\ZZ}{\mathbb{Z}}
\newcommand{\NN}{\mathbb{N}}
\newcommand{\QQ}{\mathbb{Q}}
\newcommand{\QC}{\mathbb{Q}^\complement}
\newcommand{\FF}{\mathbb{F}}
\newcommand{\Tcal}{\mathcal{T}}
\newcommand{\Mcal}{\mathcal{M}}
\newcommand{\Ncal}{\mathcal{N}}
\newcommand{\Ccal}{\mathcal{C}}
\newcommand{\Bcal}{\mathcal{B}}
\newcommand{\e}{\epsilon}
\newcommand{\ve}{\varepsilon}
\newcommand{\defeq}{\coloneqq}
\newcommand{\eqdef}{\eqqcolon}
\DeclareMathOperator{\id}{id}
\DeclareMathOperator{\sgn}{sgn}
\DeclareMathOperator{\im}{im}
\DeclareMathOperator{\mip}{mip}
\DeclareMathOperator{\Spl}{Spl}
\DeclareMathOperator{\Gal}{Gal}
\usepackage{hyperref}
\hypersetup{
  colorlinks=true,
  linkcolor=black,
  filecolor=red,      
  urlcolor=blue,
}

% \begin{lstlisting}

% \end{lstlisting} 
\lhead{\textsc{YSC3236}}
\chead{\textsc{}}
\rhead{\textsc{Oishik Ganguly A0138306J}}
\title{Final Project Part 1: Exploring the properties of 2-by-2 matrices}
\author{Oishik Ganguly}
\date{\today}

\begin{document}
\maketitle
\tableofcontents

\section {Introduction}
Matrices are well understood mathematical structures as a consequence of the
wide applicability of linear algebra to fields ranging from computer science to
economics. Basic properties and operations relating to matrices are thus familiar
even to high school students. In the first part of this final project, we consider
a specific kind of matrix, namely, the 2-by-2 matrix. In particular, we 
formalize their structure as an inductive type in coq, formalize 
the operations of matrix addition, matrix multiplication, and matrix transposition
for 2-by-2 matrices, and finally, prove certain properties about general 2-by-2 
matrices and specific 2-by-2 matrices. In this report, we will not dwell on the
formalization of the above mentioned operations:  they are unremarkable and may be viewed in the coq project file. Instead,
we will jump straight into a discussion of the properties we proved. Finally, note 
that henceforth, when we refer to a property of or operation on a matrix, we are 
referring to a 2-by-2 matrix in particular. 

\section{A note on matrix representation, operations, and paraphernalia}
Given the inductive type defintion \verb-m22- for the 2-by-2 matrices, we decided that a matrix
of the form :
$$ 
\begin{bmatrix}
  a_{11} & a_{12} \\
  a_{21} & a_{22}
\end{bmatrix}
$$
would be represented as an element of \verb-m22- as follows : 
\verb-M22 a11 a12 a21 a21-. The definitions \verb-m22_addition- ,
\verb-m22_multiplication-, \verb-m22_exponentiation-, 
\verb-m22_exponentiation_alternative-, \verb-m22_transpose- follow from
this representation of the 2-by-2 matrix. We also defined
a notion of matrix equality (\verb-A =m= B- in terms of notation) by an 
element-wise comparison of the two matrices concerned. 


\section{Associativity of matrix multiplication}

This property was captured in coq as follows: 

\begin{lstlisting}
  Proposition m22_multiplication_is_associative :
  forall (matrix_a matrix_b matrix_c : m22),
    m22_multiplication matrix_a (m22_multiplication matrix_b matrix_c) =
    m22_multiplication (m22_multiplication matrix_a matrix_b) matrix_c.
\end{lstlisting}

The proof for this property was straightforward, and mostly challenging because
of the algebraic manipulations required. We inducted over the structures of the
3 matrices, thus following the usual pen-and-paper approach to the proof of writing
out the elements of the matrices as variables. Finally, we used 
coq's \verb-do- construct to repeat a set of algebraic simplifications which gives
us the required equality.

\section{Neutrality of the identity element}

The neutrality of the identity element, captured in mathematical notation as 
$M \times I = I \times M = M$, was represented in coq via 2 lemmas, 
\verb-m22_mult_identity_r- and \verb-m22_mult_identity_l-. The proofs
for these lemmas were direct, and involved inducting over the structure of the given
matrix.


\section{Taking the n-th power of the M1 matrix}
We defined the following 2-by-2 matrix : 

$$
\begin{bmatrix}
  1 & 1 \\
  0 & 1
\end{bmatrix} 
$$

as the matrix \verb-M1- in coq, and then proved the following property about it :

\begin{lstlisting}
  Proposition about_M1 : 
  forall (n : nat),
    m22_exponentiation M1  n = M22 1 n 0 1.
\end{lstlisting}

We proved this by induction over \verb-n-. The proof itself is standard, and goes 
through using the unfold lemmas for the exponentiation operation. We employed a similar
method for the next property, which is exactly the same as above except that it
uses the alternative definition of matrix exponentiation. 

\section{ Exponentiation is commutative}

This proposition captures the following property of exponentiation of matrices : 

\begin{lstlisting}
  Proposition m22_exponentiation_is_commutative :
  forall (matrix_a : m22) (n : nat),
    m22_multiplication (matrix_a) (m22_exponentiation matrix_a n) =
    m22_multiplication (m22_exponentiation matrix_a n) (matrix_a).
\end{lstlisting}

The proof for this proceeded by induction as well. We initially attempted to prove
the property by further inducting on the structure of \verb-matrix_a-. However,
following the handwritten proof for the property, we could make use of the neutrality
of the identity element and associativity of matrix multiplication properties that
we proved earlier to see this proof through with minimal effort. Likewise,
using these tools, we saw the proof through for the next property, involving 
\verb-m22_exponentiation_alternative-, without a hitch.


\section {The equivalence of the two definitions of exponentiation}

The equivalence of the two definitions of matrix exponentiation, 
\verb-m22_exponentiation- and \verb-m22_exponentiation_alternative-, 
was proved using induction. Using the previous property concerning the commutativity
of exponentiation, this proof followed through. 

\section {Taking the n-th power of the M2 matrix}

The \verb-M2- matrix was defined as the following matrix : 

$$
\begin{bmatrix}
  1 & 0 \\
  1 & 1
\end{bmatrix}
$$

The proof for this, as with the proof for the property concerning \verb-M1-, 
uses induction, and follows through directly. The only effort required is with 
the algebraic manipulation at the end of the proof. 


\section {Transposition is involutive}
The proof for this property directly follows by inducting over the structure of
a matrix and unfolding \verb-m22_transpose- (since the operation is \textit{not}
recursively defined).

\section {Transpose of the product of matrices}

Mathematically, this property is represented is as follows :

$$ (M_1 \times M_2)^T = M_2^T \times M_1^T $$. 

The proof involved using the inductive structure of the two matrices, and then 
unfolding \verb-m22_transpose-. At this point, the equality was clear, and some
algebraic manipulation, namely, using the commutativity of multiplication of natural
numbers, needed to be used. This took up the bulk of the proof. 

\section {Transposition and exponentiation commute}

The interchangeability of the exponent and the transposition symbol for matrices 
is captured in coq as follows :

\begin{lstlisting}
Proposition m22_transposition_and_exponentiation_commute :
  forall (matrix_a : m22) (n : nat),
    m22_transpose (m22_exponentiation matrix_a n) =
    m22_exponentiation (m22_transpose matrix_a) n.
\end{lstlisting}

The proof this uses induction on \verb-n-. For the inductive case, we used the 
transposition of products property that we just proved. This brought us to the 
following stage : 

\begin{lstlisting}
m22_multiplication (m22_transpose matrix_a) (m22_exponentiation (m22_transpose matrix_a) n') =
   m22_exponentiation (m22_transpose matrix_a) (S n')
\end{lstlisting}

We realized that the left hand side corresponded to unfolding
the alternative definition of exponentiation on \verb-S n-. We thus used the
equivalence of two definitions of exponentiation, refolded the left hand side, and
finally used the equivalence on the right hand side to get an equality.


\section{Proving the n-th power of M2 property without induction}

If the coq representation does not immediately make obvious the fact that \verb-M1- 
is the transpose of \verb-M2-, the mathematical representation does : 

$
M2 = \begin{bmatrix}
  1 & 0 \\
  1 & 1
\end{bmatrix}^T = 
$ 
$
\begin{bmatrix}
  1 & 1 \\
  0 & 1
\end{bmatrix} = M1
$
. We use this simple fact and \verb-M1-'s property to prove this proposition. 

\section{ The special property of F}

Matrix \verb-F- was defined as follows: 

$$ 
\begin{bmatrix}
  1 & 1 \\
  1 & 0
\end{bmatrix}
$$

Playing around with the powers of \verb-F-, we found that it displayed properties
similar to a Fibonacci sequence. We observed that $ \forall n \geq 0, F^n$ returns a 
matrix with the n-th element of the fibonacci sequence in row 1 column 2. 
Furthermore, we found that $\forall n, F^ {n+2} = F^{n+1} + F^n$. We prove 
this in \verb-Proposition about_F-, represented as follows: 

\begin{lstlisting}
Proposition about_F :
  forall (n : nat),
    m22_addition (m22_exponentiation F n) (m22_exponentiation F (S n)) =
    m22_exponentiation F (S (S n)).
\end{lstlisting}

To prove this statement, we made use of our own induction principle, namely,  
\verb-nat_ind2-. Proving the base cases involved a great deal of algebraic 
manipulation, as evidenced by the multiple use of the \verb-do- tactic. 

For the inductive case, our strategy was as follows: to show that 
$F^{n+4} = F^{n+3} + F^{n+2}$, we factorized the right hand side to obtain the
form $(F^{n+1} + F^{n+2}) 
\cdot F$, and then rewrote the parenthesized expression with the
second induction hypotheses (namely, that $F^{n+1} + F^{n+2} = F^{n+3}$). To enable
this factorization, we also defined our distribution of matrix multiplication
lemmas before the proof for \verb-about_F-.

Finally, we proved that the element in the first row and second column of $F^n$ 
was the n-th element of the fibonacci sequence. The proposition for this
theorem was as follows :


\begin{lstlisting}
Lemma the_fibonacci_character_of_F :
  forall (n : nat),
    (match m22_exponentiation F n with
    | M22 _ a12 _ _ =>
      a12
     end) = fib_v0 n.
\end{lstlisting}

where \verb-fib_v0- is a non-accumulator implementation of the Fibonacci function. 
The proof for this proceeded by our own induction principle. For the inductive case,
we used \verb-about_F- to represent the expression being matched in the
left hand side of the goal as a sum. We then represented $F^{S n}$ and $F^{S (S n)}$ 
in terms of the inductive structures of a matrix, and the equality we required 
followed after using the inductive hypotheses. 

\section {Conclusion}

In the first section of this final project, we explored a coq based formalization 
of 2-by-2 matrices and proved properties related to the operations of addition,
multiplication, exponentiation, and transposition. We also looked at the properties
of specific matrices, namely, \verb-M1-,\verb- M2-, and \verb-F-. Over the course of 
these proofs, we gained a stronger understanding of inductive structures, inductive 
proofs, and how to make coq proofs correspond to hand written proofs. 





\end{document}

